\documentclass[a4paper,12pt,smallheadings]{scrartcl}
\usepackage[a4paper,margin=2.5cm]{geometry}
\usepackage{color,soul}
\usepackage{german}
\usepackage[utf8x]{inputenc}
\usepackage{scrpage2}
\usepackage[pdfauthor={},pdftitle={},pdfstartview=FitH,pdfborder={0 0 0}]{hyperref}
\setkomafont{pagehead}{\normalfont\normalcolor}

%\addtolength{\textheight}{5mm}
\title{HB9UF, die UHF-Gruppe der USKA}
\author{}
\date{}
\pagestyle{scrheadings}
\lohead{}
\rohead{}
\cfoot{}
\lofoot{}
\rofoot{}
\renewcommand*{\titlepagestyle}{scrheadings}
\newenvironment{q}{\small}{}
\setkomafont{pagehead}{\sf}
\sloppy
\renewcommand{\familydefault}{\sfdefault}
\begin{document}
\maketitle
\vspace{-3cm}

\section*{Über uns}
Die ``UHF-Gruppe der USKA'' ist ein Verein nach Art. 66 OR und wurde im März 1970 in
Zug von 21 OMs gegründet. Das erste 70cm Relais war bereits im Frühjahr 1967
anlässlich der USKA-GV auf dem UTO durch Gründungsmitglieder der UHF-Gruppe in
Betrieb genommen worden. Die UHF-Gruppe ist seit 2008 als Sektion
stimmberechtigtes Mitglied der USKA. Der Vorstand besteht aus bis zu 8
Funkamateuren, die ehrenamtlich arbeiten. Der Verein zählt rund 240 Mitglieder
aus der ganzen Schweiz und teilweise aus dem Ausland. Wir haben keinen Stamm –
unser Kontakt sind die QSOs über unsere Relais, unser einziges Vereinstreffen
ist die jährliche Generalversammlung.

\section*{Was machen wir?}
Wir entwickeln, testen, erstellen und betreiben Amateurfunk-Relais im 70 cm-
und 23 cm-Band und bieten dort teilweise auch EchoLink\textsuperscript{®}-Zugänge an. Wir
betreiben und moderieren im Web ein Amateurfunk-Forum, das allen Funkamateuren
zur Beantwortung von Fragen oder für Diskussionen offen steht.

\section*{Wo sind wir?}
Wir betreiben 70 cm-Relais auf dem Uetliberg/ZH, in Muttenz/BL, auf dem
Säntis/AR, auf dem Pilatus/OW, in Locarno/TI, in Winterthur/ZH und in
Zofingen/AG. Wir betreiben auch ein 23 cm-Relais auf dem Uetliberg/ZH.


\section*{Wie unterstützen?}
Wir finanzieren uns hauptsächlich über die Jahresbeiträge unserer Mitglieder
und Spenden von Funkamateuren, die unsere Anlagen benutzen. So sind wir für
beides dankbar: Neue Mitglieder und Gönner/Spender. Und jeder Funkamateur, der
unsere Anlage benutzt, verteidigt so auch die den Funkamateuren zustehenden
Frequenzen gegen die Verwendung durch andere Funkdienste. Für Arbeitseinsätze
sind auch Helfer stets willkommen. Melde dich bei Einsatzwille bitte beim
Präsidenten.

\section*{Weshalb unterstützen?}
Das Erstellen von Relais kostet Geld, der Betrieb auch. Darum sind wir auf
deine Hilfe als neues Mitglied angewiesen. Nur ein Verein mit möglichst vielen
Mitgliedern kann die Mittel, welche für den Betrieb, den Unterhalt und die
Erneuerungen der aktuell 8 Anlagen notwendig sind, bereitstellen.

\section*{}\vspace{-1cm}

Mach mit, hilf mit!

\end{document}
